%&Plain 

%\input xy
%\xyoption{all}
%\xyoption{line}
%%% Diagram for Menelaus' Theorem intersects the triangle twice.

\hbox{
\xy    <1cm,0cm>:<0cm,1cm>::
       (1.125,1.5)="A"; (-2.25,-1.5)="B"; (2.25,-1.5)="C"; 
       "A"; "B" **\dir{-}; "C" **\dir{-}; "A" **\dir{-};
	 "C"; (4.85,-1.5) **\dir{--} ?>*\dir{>};
	 (-2,1.8)="V"; (4.65,-2.25)="U" **\dir{-} ?>*\dir{>} ?<*\dir{<}; 
       {"A";"B":"U";"V",x} ="F";   {"C";"B":"U";"V",x} ="D";   {"A";"C":"U";"V",x} ="E";
       "B"+<-0.5em,-1.5ex> *{B}; 
       "C"+<0.25em,-1.5ex> *{C}; 
       "A"+<0em,1.25ex> *{A}; 
       "D"+<0.25em,1.5ex> *{D}; 
       "E"+<-0.7em,-0.75ex> *{E};  
       "F"+<-0em,1.75ex> *{F};  
       {"A";"A"+(0.81,1.33):"U";"V",x} ="X"; "A"; "X" **\dir{.};  
       {"B";"B"+(0.81,1.33):"U";"V",x} ="Y"; "B"; "Y" **\dir{.};  
       {"C";"C"+(0.81,1.33):"U";"V",x} ="Z"; "C"; "Z" **\dir{.};  
       "X"+<-0.5em,-1ex> *{X};  
           "X"+(0.1215,0.1995); "X"+(0.1215,0.1995)+(0.1995,-0.1215) **\dir{.};
           "X"+(0.1995,-0.1215) **\dir{.};
       "Y"+<0.5em,1ex> *{Y};  
           "Y"-(0.1215,0.1995); "Y"-(0.1215,0.1995)+(0.1995,-0.1215) **\dir{.};
           "Y"+(0.1995,-0.1215) **\dir{.};
       "Z"+<0.5em,1ex> *{Z};  
           "Z"-(0.1215,0.1995); "Z"-(0.1215,0.1995)-(0.1995,-0.1215) **\dir{.};
           "Z"-(0.1995,-0.1215) **\dir{.};
\endxy}


%\bye