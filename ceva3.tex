%&Plain 

%\input xy
%\xyoption{all}
%\xyoption{line}
%%% Diagram for Ceva's Theorem intersects the triangle twice.

\hbox{
\xy    <3cm,0cm>:<0cm,3cm>::
       (-1,-0.5)="B"; (1,-0.5)="C"; 
       "B"+a(40.3)="AB"; "C"+a(135.8)="AC"; {"B";"AB":"C";"AC",x}="A"; 
       "A"; "B" **\dir{-}; ?(0.42857)="N";  ?(0.21)*^!/8pt/{6}; ?(0.7)*^!/8pt/{8}; 
       "B"; "C" **\dir{-}; ?(0.75)="L"; ?(0.88)*^!/8pt/{y}; ?(0.37)*^!/8pt/{x};
       "A"; "C" **\dir{-};  ?(0.34)*_!/8pt/{9}; ?(0.88)*_!/8pt/{4};
       {"A";"L":"C";"N",x} ="P";   {"A";"C":"B";"P",x} ="M";
%
       "A"; "L" **\dir{-};
       "B"; "M" **\dir{-};
       "C"; "N" **\dir{-};
%
%
       "A"+<0em,1.25ex> *{A}; 
       "B"+<-0.5em,-1.5ex> *{B}; 
       "C"+<0.25em,-1.5ex> *{C}; 
%
       "L"+<0em,-1.25ex> *{L}; 
       "M"+<0.5em,1.5ex> *{M}; 
       "N"+<-0.25em,1.5ex> *{N}; 
%
\endxy}


%\bye


